\documentclass{skript}

\title{Global Analysis I}
\lecturer{Dr.\ Koen van den Dungen}
\author{Nicholas Schwab}
\date{Winter Term 2019/20}


\bibliography{literature}

\begin{document}
\thispagestyle{plain}
These notes are on the lecture Global Analysis I (V3B3/F4B1) in the winter term 2019/20 at the University Bonn.

\tableofcontents

\chapter{Introduction}
The main topics of this lecture include differential forms, Riemann geometry and curvature as well as de Rham cohomology. 
We strife toe proof the theorems of Stokes, Hopf-Rinow, Gauss-Bonnet and de Rham.
If we have time, we will furthermore proof the Hodge decomposition. 
The lecture will be based on two books (\cite{lee_smooth}~and~\cite{lee_riemannian}) by Lee.

Exercise classes are scheduled for Tuesdays 8:30--10:00 and 16:15--17:45 starting on the 15th of October. 
The exercise sheets are due every Monday at 10 o'clock during the lecture or in room 1.032.
Exercise groups up to four students are allowed.
The course homepage can be found on~\cite{prof_webpage}
Exercise sheets will be put on~\cite{lecture_webpage}

\chapter{Fundamentals on manifolds}
\section{Topological manifolds}

\begin{definition}[Topological manifolds]\label{def:topmanf}
    A topological space $M$ is called a \emph{topological manifold} of dimension $n$ if $M$ is
    \begin{itemize}
        \item
            Hausdorff: For all $p\neq q\in M$ there are disjoint open subsets  $U,V\subset M$ such that $p\in U$, $q\in V$.
        \item 
            second-countable, i.e.\ there exists a countable basis for the topology of $M$.
        \item
            locally Euclidean of dimension $n$: For all $p\in M$ there are open subsets $U\subset M$, $\hat U\subset \IR^n$ such that $p\in U$ and $U\simeq \hat U$.
    \end{itemize}
\end{definition}

\begin{fact}
    The dimension is a topological invariant.
\end{fact}

\begin{note}
    The empty set is a topological manifold of arbitrary dimension.
\end{note}

\begin{bsp}
    The space $\IR^n$ is a topological manifold of dimension $n$. It is Hausdorff as a metric space. The open balls with rational centers and rational radii form a countable base and we can take $U=\hat U = \IR^n$.
\end{bsp}

\begin{definition}[Coordinate chart]\label{def:coordchar}
    A \emph{coordinate chart} on $M$ is a pair $(U, \phi)$, where $U\subset M$ is open and non-empty and $\phi\colon U\to \hat U$ is a homeomorphism to an open subset $\hat U \subset \IR^n$. 
    If $\phi(U)$ is an open ball, $U$ is called a coordinate ball.
    We obtain coordinate functions $x^j$ for $j=1,\dots, n$ defined by $\phi(p) = (x^1(p), \dots, x^n(p))$.
\end{definition}

\begin{bsp}[Spheres]
    $S^n\subset \IR^{n+1}$ is Hausdorff and second-countable.
    For $j=1, \dots, n+1$, define the open subsets 
    \begin{align*}
        U^\pm_j = \left\{(x^1, \dots, x^{n+1})\in \IR^{n+1}\such \pm x^j > 0\right\}
    \end{align*}
Define $\phi^\pm_j\colon U^\pm_j \cap S^n \to B^n$ by $\phi^\pm \coloneqq (x^1, \dots, \hat{x}^j, \dots, x^{n+1})$ where $\hat{x}^j$ denotes omission of $x^j$.
    Each $\phi^\pm_j$ is a homeomorphism with inverse 
    \begin{align*}
        (u^1, \dots, u^n) \mapsto \left(u^1,\dots, \pm\sqrt{1-|U|^2}, \dots, u^n\right)
    \end{align*}
    Then we have coordinate charts $(U^\pm_j\cap S^n, \phi^\pm_j)$ which cover $S^n$.
    So $S^n$ is a topological manifold.
\end{bsp}

\begin{bsp}
    Let $M$ and $N$ be topological manifolds of dimension $m$ and $n$ respectively. 
    Then $M\times N$ is a topological manifold of dimension $m+n$. 
    Given $(p,q)\in M\times N$ pick coordinate charts $(U,\phi)$ around $p$ and $(V, \psi)$ around $q$. 
    Then $(U\times V, \phi\times\psi)$ is a coordinate chart around $(p,q)$.
\end{bsp}

\begin{cor}
    The $n$-dimensional torus $\IT^n = \prod_{i=1}^n S^1$ is a topological manifold.
\end{cor}

\section{Smooth manifolds}
\begin{note}
    Smooth means of class $C^\infty$ which means infinitely continuously differentiable.
\end{note}

\begin{definition}[Fundamental terms related to smooth manifolds]\label{def:transmap}
    Given a topological manifold and coordinate charts $(U,\phi)$ and $(V,\psi)$ such that $U$ and $V$ intersect, we have a \emph{transition map} $\psi\circ \phi^{-1}\colon \phi(U\cap V) \to \psi(U\cap V)$.
Two coordinate charts are smoothly compatible if $U\cap V = \emptyset$ or their transition map is a diffeomorphism (i.e.\ a smooth and bijective map with a smooth inverse). 

    An atlas for $M$ is a collection of coordinate charts covering $M$.
    An atlas is smooth if any two charts are smoothly compatible.
    A smooth atlas is maximal if it is not properly contained in any larger smooth atlas.
\end{definition}

\begin{definition}[Smooth manifold]\label{def:smoothmanf}
    A \emph{smooth structure} on a topological manifold $M$ is a maximal smooth atlas.
    A smooth manifold is a topological manifold with a given smooth structure.
\end{definition}

\begin{lemma}\label{lem:opensmoothmaps}
    Let $U\subseteq \IR^n$ be open and let $F\colon U\to\IR^n$ be a smooth  map whose Jacobian is non-singular at each point. 
    Then $F$ is open.
    Moreover if $F$ is injective, it is a diffeomorphism.
\end{lemma}

\begin{prop}
    \begin{alphanumerate}
        \item
            Every smooth atlas is contained in a unique maximal smooth atlas and hence determines a smooth structure.
        \item 
            Two smooth atlases determine the same smooth structure if and only if their union is again a smooth atlas.
    \end{alphanumerate}
\end{prop}
\begin{proof}
    \begin{alphanumerate}
        \item
            Given a smooth atlas $\Aa$, let $\overline \Aa$ be the set of all charts which are smoothly compatible with every chart in $\Aa$.
            Given $(U,\phi), (V,\psi)\in \overline \Aa$ with $U\cap V\neq \emptyset$, pick $x=\phi(p) \in \phi(U\cap V)$.
            Then, there exists a chart $(W,\theta)\in \Aa$ with $p\in W$.
            Therefore $\psi \circ\phi^{-1} = (\psi\circ \theta^{-1}) \circ (\theta\circ \phi^{-1})$ is smooth on a neighborhood of $x$.
            Thus $\overline \Aa$ is a smooth atlas which is clearly maximal.
            If $\Bb$ is another smooth atlas containing $\Aa$, we have $\Bb \subset\overline \Aa$, so if $\Bb$ is maximal we must have $\Bb = \overline \Aa$.
        \item 
            Is left as an exercise.
    \end{alphanumerate}
\end{proof}

\begin{bsp}
    The standard smooth structure on $\IR^n$ is the one obtained by the single chart $(\IR^n, \id)$.
    Another smooth structure on $\IR$ is determined by $\psi\colon \IR\to \IR$, $\psi(x) = x^3$.
    Here, $\psi = \psi\circ \id^{-1}$ is smooth and bijective but its inverse $\psi^{-1}(y) = y^{1/3}$ is not smooth at $y=0$.
    Hence this smooth structure is different from the standard one.
\end{bsp}

\begin{bsp}[Spheres]
    The standard smooth structure on the sphere $S^n\subset\IR^{n+1}$ is the one obtained from the atlas $\{(U^\pm_j\cap S^n, \phi^\pm_j)\}_j$.
    This atlas is indeed smooth, e.g. for $j<k$ we have 
    \begin{align*}
        \phi^\pm_j\circ (\phi_k^\pm)^{-1}(x_1,\dots, x_{n+1}) = \left(x_1,\dots, \hat x^j, \dots, \sqrt{1-|x|^2}, \dots, x^{n+1}\right)
    \end{align*}
    which is smooth  on $B^n$.
\end{bsp}

\begin{lemma}
    Let $M$ be a set with a collection of subsets $\{U_\alpha\}$ of $M$ and maps $\phi_\alpha\colon U_\alpha \to \IR^n$ such that
    \begin{enumerate}
        \item 
            each $\phi_\alpha\colon U_\alpha\to \phi(U_\alpha)$ is bijective, and $\phi_\alpha(U_\alpha)\subset \IR^n$ is open.
        \item
            for all $\alpha, \beta$ the sets $\phi_\alpha(U_\alpha\cap U_\beta)$ and $\phi_\beta(U_\alpha\cap U_\beta)$ are open.
        \item
            If $U_\alpha\cap U_\beta \neq \emptyset$, the map $\phi_\beta\circ\phi_\alpha^{-1}\colon \phi_\alpha(U\alpha\cap U_\beta) \to \phi(U_\alpha\cap U_\beta)$ is smooth.
        \item
            $M$ can be covered by countably many $U_\alpha$
        \item
            for $p\neq q\in M$, either $p,q\in U_\alpha$ for some $\alpha$ or there exist disjoint $U_\alpha$ and $U_\beta$ with $p\in U_\alpha$ and $q\in U_\beta$.
    \end{enumerate}
    Then $M$ has a unique smooth manifold structure such that each $(U_\alpha, \phi_\alpha)$ is a smooth chart.
\end{lemma}
\begin{proof}
    Idea: Define the topology by the basis $\left\{\phi_\alpha^{-1}(V)\right\}$ for $V\subset\IR^n$ open.
\end{proof}

\section{Manifolds with boundary}
There look locally like either $\IR^n$ or the closed upper half-space 
\begin{align*}
    \IH^n\coloneqq \left\{(x^1,\dots, x^n)\in \IR^n: x^n\geq 0\right\}\;.
\end{align*}
The interior and boundary are denoted by
\begin{align*}
    \Int \IH^n   &= \left\{(x^1,\dots, x^n) \in \IR^n \such x^n > 0\right\}\\
    \partial \IH^n &= \left\{(x^1,\dots, x^n) \in \IR^n \such x^n = 0\right\}\;.
\end{align*}

\begin{definition}[Topological manifold with boundary]\label{def:topmanfbound}
    A topological manifold with boundary is a second-countable Hausdorff space $M$ in which every point has a neighborhood homeomorphic to an open subset in $\IR^n$ or to a open subset in $\IH^n$.

A chart $(U,\phi)$ is called 
\begin{itemize}
    \item 
        an interior chart if $\phi(U) \subset \IR^n$ is open.
    \item
        a boundary chart if $\phi(U)\subset \IH^n$ is open and $\phi(U) \cap \partial\IH^n \neq \emptyset$.
\end{itemize}
A point is called
\begin{itemize}
\item
    an interior point if it is in the domain of an interior chart. 
    The interior points form the set $\Int M$.
\item 
    a boundary point if it is in the domain of a boundary chart that sends $p$ to $\partial \IH^n$. 
    The boundary points form the set $\partial M$.
\end{itemize}
If $U\subset \IH^n$ is open, $F:U\to \IR^k$ is smooth if for all $x\in U$ such that there is an open subset $\hat U\subset \IR^n$ with $x\in \hat U$ and a smooth map $\hat F\colon \hat U \to \IR^k$ such that $F\equiv \hat F$ on $U\cap \hat U$.
We then define a smooth structure on manifolds with boundary in the same manner as for topological manifolds.

\begin{thm}
    Let $M$ be a smooth manifold with a boundary and let $p\in M$. 
    If there is a smooth chart $(U, \phi)$ around $p$ such that $\phi(U) \subseteq \IH^n$ and $\phi(p) \in \partial\IH^n$, then the same is true for every smooth chart.
    Thus $M= \Int M \sqcup \partial M$.
\end{thm}
\begin{proof}
    Suppose the contrary that there is also smooth chart interior chart $(V,\psi)$ around $p$.
    Then we have a diffeomorphism 
    \begin{align*}
        \tau \coloneqq \phi\circ \psi^{-1} \colon \psi(U\cap V) \longrightarrow \phi(U\cap V)
    \end{align*}
    Write $x_0 = \psi(p)$ and $y_0 = \phi(p) = \tau(x_0)$. 
    There is a neighborhood $W$ of $y_0$ in $\IR^n$ and a smooth $\eta\colon W\to \IR^n$ such that $\eta \equiv \tau^{-1}$ on $W\cap \phi(U\cap V)$. 
    Since $(V,\psi)$ is interior $\tau$ is smooth on an open ball $B\subset \psi(U\cap V)$ around $x_0$.
    Without loss of generality, we may assume $B\subset \tau^{-1} (W)$. 
    Then $\eta\circ \tau|_B = \id_B$ and by chain rule $D\eta(\tau(x)) \cdot D\tau(x)$ is the identity matrix for each $x\in B$.
    In particular $D\tau$ must be non-singular.
    Hence, $\tau \colon B\to \IR^n$ is open.
    Thus $\tau(B)\subset \IR^n$ is open and $y_0\in \tau(B)\subset \phi(U)$ which implies $y_0\not\in \partial \IH^n$. 
\end{proof}

\end{definition}

\section{Smooth maps}
\begin{definition}[Smooth map]\label{def:smoothmap}
Let $M,N$ be smooth manifolds, with or without boundary.
A map $F\colon M\to N$ is \emph{smooth} if for every $p\in M$ there exist smooth charts $(U,\phi)$ around $p$ and $(V,\psi)$ around $F(p)$ such that $F(U)\subset V$ and $\psi\circ F\circ \phi^{-1}\colon \phi(U) \to \psi(V)$ is smooth. 
\end{definition}

\begin{prop}\label{prop:smoothmapcont}
    Every smooth map is continuous.
\end{prop}
\begin{proof}
    Write $F|_U = \psi^{-1}\circ (\psi\circ F\circ \phi^{-1})\circ \phi$, 
    where $\psi^{-1},\phi$ are homeomorphisms, and $\psi\circ F\circ\phi^{-1}$ is smooth on $\IR^n$ (hence continuous).
\end{proof}

\begin{prop}
    For a map $F\colon M\to N$, the following are equivalent:
    \begin{alphanumerate}
    \item
        $F$ is smooth.
    \item
        $\forall p\in M$ there are smooth charts $(U,\phi)$ around $p$ and $(V,\psi)$ around $F(p)$ such that $U\cap F^{-1}(V)$ is open in $M$ and $\psi\circ F\circ \phi^{-1} (U\cap F^{-1}(V)\to \psi(V)$ is smooth.
    \item
        $F$ is continuous and there exist smooth atlases $\{(U_\alpha, \phi_\alpha)\}$ for $M$ and $\{(V_\beta, \psi_\beta)\}$ for $N$ such that $\forall \alpha,\beta$ the map
        \begin{align*}
            \psi_\beta \circ F \circ \phi^{-1}_\alpha (U_\alpha\cap F^{-1}(V_\beta)) \to \psi_\beta (V_\beta)
        \end{align*}
        is smooth.
    \end{alphanumerate}
    
\end{prop}

\begin{prop}
    Consider a map $F\colon M\to N$.
    \begin{alphanumerate}
    \item 
        If every $p\in M$ has a neighborhood $U$ such that $F|_U$ is smooth, then $F$ is smooth.
    \item
        If $F$ is smooth, then $F|_U$ is smooth for every $U\subset M$.
    \end{alphanumerate}
    
\end{prop}

\begin{cor}
    Let $\{U_\alpha\}$ be an open cover of $M$.
    Suppose we have smooth maps $F_\alpha\colon U_\alpha \to N$ such that $F_\alpha \equiv F_\beta$ on $U_\alpha\cap U_\beta$.
    Then there exists a unique smooth map $F\colon M\to N$ such that $F\equiv F_\alpha$ on $U_\alpha$ for each $\alpha$.
\end{cor}

\begin{prop}
    \begin{alphanumerate}
    \item 
        Every constant map is smooth.
    \item
        The identity map is smooth.
    \item
        For any open submanifold $U\subset M$, the inclusion map $U\hookrightarrow M$ is smooth.
    \item
        The composition of two smooth maps is again smooth.
    \end{alphanumerate}
\end{prop}

\begin{definition}[Diffeomorphism]\label{def:diffeo}
    A \emph{diffeomorphism} from $M$ to $N$ is a smooth bijective map that has a smooth inverse.
    If there exists a diffeomorphism, we say $M$ and $N$ are diffeomorphic denoted by $M \approx N$.
\end{definition}

\begin{bsp}
    Any smooth chart $(U,\phi)$ gives a diffeomorphism $\phi\colon U \to \phi(U)$.
\end{bsp}

\begin{prop}
    \begin{alphanumerate}
    \item
        The composition of two diffeomorphisms is again a diffeomorphism.
    \item
        A finite product of diffeomorphism is a diffeomorphism.
    \item
        Every diffeomorphism is also a homeomorphism and an open map.
    \item
        The restriction of a diffeomorphism to an open subset is a diffeomorphism from this open subset to its image.
    \item
        Diffeomorphic is an equivalence relation.
    \end{alphanumerate}
\end{prop}

\begin{thm}
    If $M$ and $N$ are non-empty smooth manifolds and $M\approx N$, then $\dim M = \dim N$.
\end{thm}
\begin{proof}
    Pick $p\in M$ and charts $(U,\phi)$ around $p$ and $(V,\psi)$ around $F(p)$, where $F\colon M\to N$ is a diffeomorphism. 
    Then (the restriction of) $\hat F = \psi\circ F\circ \phi^{-1}$ is a diffeomorphism between open subsets of $\IR^{\dim M}$  and $\IR^{\dim N}$.
    Since $\hat F^{-1} \circ \hat F$ and $\hat F\circ\hat F^{-1}$ are the identity map, it follows from the chain rule that $D\hat F(\phi(p))$ is invertible.
    Hence it is square, so $\dim M = \dim N$.
\end{proof}

\begin{thm}
    If $F\colon M\to N$ is a diffeomorphism between manifolds with boundary, then $F(\partial M) = \partial N$ and $F|_{\Int M}$ is a diffeomorphism.
\end{thm}

\begin{bsp}
    Let $\IR$ be the real line with the standard smooth structure and $\tilde \IR$ be the real line with the smooth structure given by the single global chart $(\IR, \phi\colon x\to x^3)$. 
    These are different smooth structures. 
    Nevertheless $\IR \approx \tilde\IR$.

    Consider $F\colon \IR\to\tilde\IR$ given by $F(x) = x^{1/3}$. 
    Its coordinate representation is $\hat F(t) = \phi\circ F\circ \id^{-1}(t) = t$. 
    Similarly $\hat F^{-1}$ is also smooth. 
    Thus $F$ is a diffeomorphism.
\end{bsp}

\begin{fact}
    There is only one smooth structure on the real line up to diffeomorphism.
    The same is true for any topological manifold of dimension at most $3$.
\end{fact}

\section{Topological properties of manifolds}
\begin{lemma}
    Every topological manifold has a countable basis of precompact (i.e.\ whose closure is compact) coordinate balls.
\end{lemma}

\begin{cor}
    Every topological manifold is locally compact.
\end{cor}

\begin{prop}
    Let $M$ be a topological manifold.
    \begin{alphanumerate}
    \item 
        $M$ is locally path-connected (i.e.\ it has a basis of path-connected open subsets).
    \item
        $M$ is connected iff $M$ is path-connected, and components are path-components.
    \item
        $M$ has at most countably many components, and each component is an open subset of $M$ and a connected topological manifold.
    \end{alphanumerate}
\end{prop}

\begin{definition}[Paracompactness]
    \begin{alphanumerate}
    \item
        A collection $\Cc$ of subsets of $M$ is called \emph{locally finite} if each point has a neighborhood which intersects with at most finitely many of these subsets.
    \item
        Given a cover $\Uu$ of $M$, a cover $\Vv$ of $M$ is called a \emph{refinement} of $\Uu$ if for each open subset $V\in \Vv$ there is a $U\in \Uu$ such that $V\subset U$.
    \item
        $M$ is \emph{paracompact} if every open cover of $M$ has a locally finite open refinement.
    \end{alphanumerate}
    
\end{definition}
    
\begin{lemma}
    Let $\Cc$ be a locally finite collection of subsets of $M$. 
    \begin{alphanumerate}
    \item
        The collection $\left\{\overline C\such C\in\Cc\right\}$ is also locally finite.
    \item
        $\overline{\bigcup_{C\in \Cc} C} = \bigcup_{C\in \Cc}\overline C$.
    \end{alphanumerate}
    
\end{lemma}
    
\begin{thm}
    Every topological manifold is paracompact.
    In fact: Given an open cover $\Uu$ of $M$ and a basis $\Bb$ for the topology of $M$, there exists a countable, locally finite, open refinement of $\Uu$ consisting of elements of $\Bb$.
\end{thm}
\begin{proof}
    Idea: Uses that a locally compact, second-countable Hausdorff space admits an exhaustion by compact sets (i.e.\ a ascending sequence of compact sets such that the space is the limit of this sequence).
\end{proof}
Now let $M$ be a smooth manifold.
\begin{definition}
    $B\subset M$ is a \emph{regular coordinate ball} if there are a smooth chart $(B',\phi)$ and numbers $0<r<r'$ such that $B\subset B'$ and $\phi(B) = B_r(0)$, $\phi(\overline B) = \overline{B_r(0)}$ and $\phi(B') = B_{r'}(0)$.
\end{definition}

\begin{prop}
    Every smooth manifold has a countable basis of regular coordinate balls.
\end{prop}

\begin{lemma}
    \begin{alphanumerate}
    \item
        The function $f\colon \IR\to \IR$ given by $f(t)\coloneqq e^{-1/t}$ for $t>0$ and $0$ otherwise is smooth.
    \item
        Given $r_1<r_2\in \IR$, there exists a smooth function $h\colon \IR\to \IR$ such that $h(t) \equiv 1$ for $t\leq r_1$, $0<h(t)<1$ for $r_1<t<r_2$, and $h(t) \equiv 0$ for $t\geq r_2$.
    \item
        There exists a smooth function $H\colon \IR^n\to\IR$ such that $H\equiv 1$ on $\overline{B_{r_1}(0)}$, $0 < H < 1$ on $B_{r_2}\setminus \overline{B_{r_1}}$ and $H\equiv 0$ on $\IR^n \setminus B_{r_2}(0)$.
    \end{alphanumerate}
\end{lemma}

\begin{proof}
    \begin{alphanumerate}
    \item
        Compute derivatives.
    \item
        \begin{align*}
            h(t) = \frac{f(r_2-t)}{f(r_2-t) + f(t-r_1)}
        \end{align*}
    \item
        $H(x) = h(|x|)$
        
    \end{alphanumerate}
    
\end{proof}

We call $h$ a \emph{cutoff function} and $H$ a \emph{bump function}.

Recall: For a topological space $X$, and a map $f\colon X\to \IR^k$ we define the support of $f$ as $\supp f\coloneqq \overline{\left\{p\in X\such f(p) \neq 0\right\}}$.

Let $\Uu=\left\{U_\alpha\right\}_{\alpha\in A}$ be an open cover of $X$.
\begin{definition}[Partition of unity]
    A \emph{partition of unity} subordinate to $\Uu$ is a family $\{\phi_\alpha\}_{\alpha\in A}$ of continuous functions $\phi_\alpha\colon X\to \IR$ such that
    \begin{itemize}
        \item
        $0\leq \phi_\alpha(x) \leq 1$ for all $\alpha\in A, x\in X$.
    \item
        $\supp(\phi_\alpha) \subset U_\alpha$
    \item
        the family of supports $\{\supp \phi_\alpha\}$ is locally finite.
    \item 
        $\sum_{\alpha\in A} \phi_\alpha(x) = 1$ for all $x\in X$. Note: this is a finite sum by the previous property.

    \end{itemize}
    If $M$ is a smooth manifold, a partition of unity $\{\phi_\alpha\}$ is smooth $\phi_\alpha$ is smooth.
\end{definition}





        



\printbibliography
\end{document}
