\documentclass{skript}

\title{Global Analysis I}
\lecturer{Dr.\ Koen van den Dungen}
\author{Nicholas Schwab}
\date{Winter Term 2019/20}


\bibliography{literature}

\begin{document}
\thispagestyle{plain}
These notes are on the lecture Global Analysis I (V3B3/F4B1) in the winter term 2019/20 at the University Bonn.

\tableofcontents

\chapter{Introduction}
The main topics of this lecture include differential forms, Riemann geometry and curvature as well as de Rham cohomology. 
We strife toe proof the theorems of Stokes, Hopf-Rinow, Gauss-Bonnet and de Rham.
If we have time, we will furthermore proof the Hodge decomposition. 
The lecture will be based on two books (\cite{lee_smooth}~and~\cite{lee_riemannian}) by Lee.

Exercise classes are scheduled for Tuesdays 8:30--10:00 and 16:15--17:45 starting on the 15th of October. 
The exercise sheets are due every Monday at 10 o'clock during the lecture or in room 1.032.
Exercise groups up to four students are allowed.
The course homepage can be found on~\cite{prof_webpage}
Exercise sheets will be put on~\cite{lecture_webpage}

\chapter{Fundamentals on manifolds}
\section{Topological manifolds}

\begin{definition}[Topological manifolds]\label{def:topmanf}
    A topological space $M$ is called a \emph{topological manifold} of dimension $n$ if $M$ is
    \begin{itemize}
        \item
            Hausdorff: For all $p\neq q\in M$ there are disjoint open subsets  $U,V\subset M$ such that $p\in U$, $q\in V$.
        \item 
            second-countable, i.e.\ there exists a countable basis for the topology of $M$.
        \item
            locally Euclidean of dimension $n$: For all $p\in M$ there are open subsets $U\subset M$, $\hat U\subset \IR^n$ such that $p\in U$ and $U\simeq \hat U$.
    \end{itemize}
\end{definition}

\begin{fact}
    The dimension is a topological invariant.
\end{fact}

\begin{note}
    The empty set is a topological manifold of arbitrary dimension.
\end{note}

\begin{bsp}
    The space $\IR^n$ is a topological manifold of dimension $n$. It is Hausdorff as a metric space. The open balls with rational centers and rational radii form a countable base and we can take $U=\hat U = \IR^n$.
\end{bsp}

\begin{definition}[Coordinate chart]\label{def:coordchar}
    A \emph{coordinate chart} on $M$ is a pair $(U, \phi)$, where $U\subset M$ is open and non-empty and $\phi\colon U\to \hat U$ is a homeomorphism to an open subset $\hat U \subset \IR^n$. 
    If $\phi(U)$ is an open ball, $U$ is called a coordinate ball.
    We obtain coordinate functions $x^j$ for $j=1,\dots, n$ defined by $\phi(p) = (x^1(p), \dots, x^n(p))$.
\end{definition}

\begin{bsp}[Spheres]
    $S^n\subset \IR^{n+1}$ is Hausdorff and second-countable.
    For $j=1, \dots, n+1$, define the open subsets 
    \begin{align*}
        U^\pm_j = \left\{(x^1, \dots, x^{n+1})\in \IR^{n+1}\such \pm x^j > 0\right\}
    \end{align*}
Define $\phi^\pm_j\colon U^\pm_j \cap S^n \to B^n$ by $\phi^\pm \coloneqq (x^1, \dots, \hat{x}^j, \dots, x^{n+1})$ where $\hat{x}^j$ denotes omission of $x^j$.
    Each $\phi^\pm_j$ is a homeomorphism with inverse 
    \begin{align*}
        (u^1, \dots, u^n) \mapsto \left(u^1,\dots, \pm\sqrt{1-|U|^2}, \dots, u^n\right)
    \end{align*}
    Then we have coordinate charts $(U^\pm_j\cap S^n, \phi^\pm_j)$ which cover $S^n$.
    So $S^n$ is a topological manifold.
\end{bsp}

\begin{bsp}
    Let $M$ and $N$ be topological manifolds of dimension $m$ and $n$ respectively. 
    Then $M\times N$ is a topological manifold of dimension $m+n$. 
    Given $(p,q)\in M\times N$ pick coordinate charts $(U,\phi)$ around $p$ and $(V, \psi)$ around $q$. 
    Then $(U\times V, \phi\times\psi)$ is a coordinate chart around $(p,q)$.
\end{bsp}

\begin{corr}
    The $n$-dimensional torus $\IT^n = \prod_{i=1}^n S^1$ is a topological manifold.
\end{corr}

\section{Smooth manifolds}
\begin{note}
    Smooth means of class $C^\infty$ which means infinitely continuously differentiable.
\end{note}

\begin{definition}[Fundamental terms related to smooth manifolds]\label{def:transmap}
    Given a topological manifold and coordinate charts $(U,\phi)$ and $(V,\psi)$ such that $U$ and $V$ intersect, we have a \emph{transition map} $\psi\circ \phi^{-1}\colon \phi(U\cap V) \to \psi(U\cap V)$.
Two coordinate charts are smoothly compatible if $U\cap V = \emptyset$ or their transition map is a diffeomorphism (i.e.\ a smooth and bijective map with a smooth inverse). 

    An atlas for $M$ is a collection of coordinate charts covering $M$.
    An atlas is smooth if any two charts are smoothly compatible.
    A smooth atlas is maximal if it is not properly contained in any larger smooth atlas.
\end{definition}

\begin{definition}[Smooth manifold]\label{def:smoothmanf}
    A \emph{smooth structure} on a topological manifold $M$ is a maximal smooth atlas.
    A smooth manifold is a topological manifold with a given smooth structure.
\end{definition}

\begin{lemma}\label{lem:opensmoothmaps}
    Let $U\subseteq \IR^n$ be open and let $F\colon U\to\IR^n$ be a smooth  map whose Jacobian is non-singular at each point. 
    Then $F$ is open.
    Moreover if $F$ is injective, it is a diffeomorphism.
\end{lemma}

\begin{prop}
    \begin{alphanumerate}
        \item
            Every smooth atlas is contained in a unique maximal smooth atlas and hence determines a smooth structure.
        \item 
            Two smooth atlases determine the same smooth structure if and only if their union is again a smooth atlas.
    \end{alphanumerate}
\end{prop}
\begin{proof}
    \begin{alphanumerate}
        \item
            Given a smooth atlas $\Aa$, let $\overline \Aa$ be the set of all charts which are smoothly compatible with every chart in $\Aa$.
            Given $(U,\phi), (V,\psi)\in \overline \Aa$ with $U\cap V\neq \emptyset$, pick $x=\phi(p) \in \phi(U\cap V)$.
            Then, there exists a chart $(W,\theta)\in \Aa$ with $p\in W$.
            Therefore $\psi \circ\phi^{-1} = (\psi\circ \theta^{-1}) \circ (\theta\circ \phi^{-1})$ is smooth on a neighborhood of $x$.
            Thus $\overline \Aa$ is a smooth atlas which is clearly maximal.
            If $\Bb$ is another smooth atlas containing $\Aa$, we have $\Bb \subset\overline \Aa$, so if $\Bb$ is maximal we must have $\Bb = \overline \Aa$.
        \item 
            Is left as an exercise.
    \end{alphanumerate}
\end{proof}

\begin{bsp}
    The standard smooth structure on $\IR^n$ is the one obtained by the single chart $(\IR^n, \id)$.
    Another smooth structure on $\IR$ is determined by $\psi\colon \IR\to \IR$, $\psi(x) = x^3$.
    Here, $\psi = \psi\circ \id^{-1}$ is smooth and bijective but its inverse $\psi^{-1}(y) = y^{1/3}$ is not smooth at $y=0$.
    Hence this smooth structure is different from the standard one.
\end{bsp}

\begin{bsp}[Spheres]
    The standard smooth structure on the sphere $S^n\subset\IR^{n+1}$ is the one obtained from the atlas $\{(U^\pm_j\cap S^n, \phi^\pm_j)\}_j$.
    This atlas is indeed smooth, e.g. for $j<k$ we have 
    \begin{align*}
        \phi^\pm_j\circ (\phi_k^\pm)^{-1}(x_1,\dots, x_{n+1}) = \left(x_1,\dots, \hat x^j, \dots, \sqrt{1-|x|^2}, \dots, x^{n+1}\right)
    \end{align*}
    which is smooth  on $B^n$.
\end{bsp}

\begin{lemma}
    Let $M$ be a set with a collection of subsets $\{U_\alpha\}$ of $M$ and maps $\phi_\alpha\colon U_\alpha \to \IR^n$ such that
    \begin{enumerate}
        \item 
            each $\phi_\alpha\colon U_\alpha\to \phi(U_\alpha)$ is bijective, and $\phi_\alpha(U_\alpha)\subset \IR^n$ is open.
        \item
            for all $\alpha, \beta$ the sets $\phi_\alpha(U_\alpha\cap U_\beta)$ and $\phi_\beta(U_\alpha\cap U_\beta)$ are open.
        \item
            If $U_\alpha\cap U_\beta \neq \emptyset$, the map $\phi_\beta\circ\phi_\alpha^{-1}\colon \phi_\alpha(U\alpha\cap U_\beta) \to \phi(U_\alpha\cap U_\beta)$ is smooth.
        \item
            $M$ can be covered by countably many $U_\alpha$
        \item
            for $p\neq q\in M$, either $p,q\in U_\alpha$ for some $\alpha$ or there exist disjoint $U_\alpha$ and $U_\beta$ with $p\in U_\alpha$ and $q\in U_\beta$.
    \end{enumerate}
    Then $M$ has a unique smooth manifold structure such that each $(U_\alpha, \phi_\alpha)$ is a smooth chart.
\end{lemma}
\begin{proof}
    Idea: Define the topology by the basis $\left\{\phi_\alpha^{-1}(V)\right\}$ for $V\subset\IR^n$ open.
\end{proof}

\section{Manifolds with boundary}
There look locally like either $\IR^n$ or the closed upper half-space 
\begin{align*}
    \IH^n\coloneqq \left\{(x^1,\dots, x^n)\in \IR^n: x^n\geq 0\right\}\;.
\end{align*}
The interior and boundary are denoted by
\begin{align*}
    \Int \IH^n   &= \left\{(x^1,\dots, x^n) \in \IR^n \such x^n > 0\right\}\\
    \delta \IH^n &= \left\{(x^1,\dots, x^n) \in \IR^n \such x^n = 0\right\}\;.
\end{align*}

\begin{definition}[Topological manifold with boundary]\label{def:topmanfbound}
    A topological manifold with boundary is a second-countable Hausdorff space $M$ in which every point has a neighborhood homeomorphic to an open subset in $\IR^n$ or to a open subset in $\IH^n$.

A chart $(U,\phi)$ is called 
\begin{itemize}
    \item 
        an interior chart if $\phi(U) \subset \IR^n$ is open.
    \item
        a boundary chart if $\phi(U)\subset \IH^n$ is open and $\phi(U) \cap \delta\IH^n \neq \emptyset$.
\end{itemize}
A point is called
\begin{itemize}
\item
    an interior point if it is in the domain of an interior map.
\item 
    a boundary point if it is in the domain of a boundary point that sends $p$ to $\delta \IH^n$.
\end{itemize}
If $U\subset \IH^n$ is open, $F:U\to \IR^k$ is smooth if for all $x\in U$ such that there is an open subset $\hat U\subset \IR^n$ with $x\in \hat U$ and a smooth map $\hat F\colon \hat U \to \IR^k$ such that $F\equiv \hat F$ on $U\cap \hat U$.
We then define a smooth structure on manifolds with boundary in the same manner as for topological manifolds.

\end{definition}














\printbibliography
\end{document}
